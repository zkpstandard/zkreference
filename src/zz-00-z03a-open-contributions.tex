\presection{Open to contributions}
\label{sec:prelim:open-to-contributions}

	\revblock[rev:add-open-to-contributions]{\ref{it:editorial:open-to-contributions}}
	This document is open to contributions from the community. 
	The following notes provide guidance for the expected contributions.


\textbf{Purpose.}
	The purpose of developing the ZKProof Community Reference document is to provide, 
within the principles laid out by the \hyperref[sec:prelim:charter]{ZKProof charter}, 
a reference for the development of zero-knowledge-proof technology that is secure, practical and interoperable.


\textbf{Aims.}
	The aim of the document is to consolidate reference material developed and/or discussed in collaborative processes during the ZKProof workshops. 
	The document intends to be accessible to a large audience, including the general public, the media, the industry, developers and cryptographers.


\textbf{Scope.}
	The document intends to cover material relevant for its purpose --- the development of secure, practical and interoperable technology.
	The document can also elaborate on introductory concepts or works, to enable an easier understanding of more advanced techniques. 
	When a focus is chosen from several alternative options, the document should include a rationale describing comparative advantages, disadvantages and applicability. 
	However, the document does not intend to be a thorough survey about ZKPs, and does not need to cover every conceivable scenario.


\textbf{Format.}
	To achieve its accessibility goal, and considering its wide scope, the document favors the inclusion of: 
	a well defined structure (e.g., chapters, sections, subsections);
	introductory descriptions (e.g., an executive summary and one introduction per chapter); 
	illustrative examples covering the main concepts; 
	enumerated recommendations and requirements; 
	summarizing tables; 
	glossary of technical terms; 
	appropriate references for presented claims and results.


\textbf{Editorial methodology.}
	The development process of this community reference is proposed to happen in cycles of four phases:
	(i) \textbf{open discussion} during ZKProof workshops, with guided discussions and corresponding annotations to serve as reference for subsequent development;
	(ii) \textbf{content development}, according to a set of proposed contributions, during a defined period for their development by voluntary \emph{contributors};
	(iii) \textbf{integration}, by the \emph{editors}, of the obtained content and feedback into the actual document; 
	(iv) \textbf{public feedback} on the state of development of the document.
	A team of editors coordinates the process, facilitating the editorial quality of the document,
and promoting transparency by means of public calls for contributions and feedback, 
and enabling an easy way to identify the changes in the document and their rationale.

