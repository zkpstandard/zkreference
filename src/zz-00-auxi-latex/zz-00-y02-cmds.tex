%%%%%%%%%%%%%%%%%%%%%%%%%%%%%%%%%%%%%%%%%%%%%%%%%%%%%%%%%%%%%%%%
%%% This file: defines counters and commands useful for cross-referencing feedback notes and corresponding changes

\newcommand{\ifoddeven}[2]{\ifthenelse{\isodd{\value{page}}}{#1}{#2}}

\providecommand{\ifEmpty}[2]{\ifthenelse{\equal{#1}{}}{\protect #2}{}}
\providecommand{\ifEmptyElse}[3]{\ifthenelse{\equal{#1}{}}{\protect #2}{\protect #3}}
\providecommand{\ifNotEmpty}[2]{\ifthenelse{\equal{#1}{}}{}{\protect #2}}
\providecommand{\ifNotEmptyElse}[3]{\ifthenelse{\equal{#1}{}}{\protect #3}{\protect #2}}
\providecommand{\ifNotEmptyPrepend}[2]{\ifthenelse{\equal{#1}{}}{}{#2#1}}
\providecommand{\ifNotEmptyAppend}[2]{\ifthenelse{\equal{#1}{}}{}{#1#2}}


%%%%%%%%%%%% Commands
\newcommand{\tops}[2]{\texorpdfstring{#1}{#2}}
\newcommand{\nolink}[1]{\href*{#1}{#1}}

\newcommand{\subtabL}[1]{\subtab[l]{#1}}
\newcommand{\edsubtab}[2][c]{\begin{edtable}{tabular}{@{}#1}#2\end{edtable}}

\newcommand{\brkttext}[1]{\ensuremath{\left\langle\right.}#1\ensuremath{\left.\right\rangle}}
\newcommand{\brktmath}[1]{\ensuremath{\left\langle\right.#1\left.\right\rangle}}

\newenvironment{bulletize}{\begin{itemize}[label={$\bullet$}]}{\end{itemize}}


\newcommand{\myurl}[1]{\href{#1}{\nolinkurl{#1}}}
\newcommand{\myemail}[1]{\href{mailto:#1}{#1}}


%%%%%%%%%%%% Auxi commands for longtable
\newcommand{\begmin}[2][t]{\begin{minipage}[#1]{#2\textwidth}\vspace{0pt}} %-1.5ex need to coordinate with the space in last column
\newcommand{\myendmini}{\vspace{1ex}\end{minipage}}
\newcommand{\rowend}{\end{minipage}\\}


%LB: colors for bookmarks of table and Figures
\newcommand{\red}[1]{{\color[rgb]{1,0,0}#1}}
\newcommand{\blue}[1]{{\color[rgb]{0,0,1}#1}}
\newcommand{\colorbkmtab}{brown}
\newcommand{\colorbkmfig}{blue}
\definecolor{colorRowHead}{rgb}{0.95,0.95,0.95}


%%%%%%%%%%%%%%%%%%%%%%%%%%%% Caption for Table, adding special pdf bookmark and allowing line number

\newcommand{\conditionalinternallinenumbers}{\ifbool{boolShowLineNumbers}{\internallinenumbers}{}}
\let\CIL\conditionalinternallinenumbers


%%% LB: \mytabcap: to be used (instead of \caption) inside environment table (for easier line numbering)
  % does not require the `table' environment, but can be inside of it
  % Enables correct line numbers and it also creates a pdfbookmark
  % 1st arg is the text for the listoffigures (lot); 2nd arg is the main text of the caption
\newcommand{\vertspaceintabcap}{\vspace{.5em}}  %may be locally redefined 
\newcommand{\setlinenumbermargin}[1]{\ifbool{boolShowLineNumbers}{\renewcommand{\thelinenumber}{\arabic{linenumber}\hspace*{#1}}}{}} %to be used inside local context, e.g., within \begingroup...\endgroup
\newcommand{\setlinenumberintable}{}  %%% if boolShowFeedback, it will be redefined to force disappearance of line numbers


\newcommand{\mytabcap}[3][]{%
	\begingroup\CIL\normalsize\centering{\refstepcounter{table}\label{#1}\bfseries\hypertarget{ht:table:\thetable}{Table}~\thetable{}:} %intentional space before comment %
				\addcontentsline{lot}{table}{Table~\thetable{}: #2}#3\endgroup

	\vertspaceintabcap\setlinenumberintable%
				{\bookmarksetup{color=\colorbkmtab}%
	\ifthenelse{\equal{\arabic{section}}{0}}{\bookmarksetup{level=1}}{\ifthenelse{\equal{\arabic{subsection}}{0}}{\bookmarksetup{level=2}}{\ifthenelse{\equal{\arabic{subsubsection}}{0}}{\bookmarksetup{level=3}}{\bookmarksetup{level=4}}}}%
	%if wanting the pdf-bookmark to go inside the subsection, then in the line above change the last bookmarksetup-level to 3 
	\bookmark[dest=ht:table:\thetable]{Table~\thetable}%
	\bookmarksetup{color=black}%
	}
}

\newcommandtwoopt{\minip}[3][\textwidth][\textwidth]{\begingroup\nolinenumbers\resizebox{#1}{!}{\begin{minipage}[t]{#2}\conditionalinternallinenumbers #3\end{minipage}}\vskip0pt\endgroup}

%%%%%%%%%%%%%%%%%%%%%%%% Caption for Figure, adding special pdf bookmark

\newcommand{\myfigcap}[2]{\refstepcounter{figure}\centering{\bfseries Figure~\thefigure{}:}
														 \addcontentsline{lof}{figure}{Figure~\thefigure{}: #1} #2 \nolinenumbers}


%%%%%%%%%%%%%%%%%%%%%%%%%%%%%%%%%%%%%%%%%%%%%%%%%%%%%%%%%%%%%%%%
%%%%%% Commands handling hyper-references

\urlstyle{rm} %\renewcommand\UrlFont{\color{red}\rmfamily\itshape}  %tm, tt, sf, ... same, ...

\let\incgr\includegraphics
\newcommand{\mcc}[2][1]{\multicolumn{#1}{c}{#2}}
\newcommand{\restoline}[2][1.]{\resizebox{#1\linewidth}{!}{#2}}
\newcommand{\refchap}[1]{Chapter~\ref{#1}}
\newcommand{\refsec}[1]{Section~\ref{#1}}
\newcommand{\reftab}[1]{Table~\ref{#1}}
\newcommand{\reffig}[1]{Fig.~\ref{#1}}
\newcommand{\reffigure}[1]{Figure~\ref{#1}}
\newcommand{\refsecnamenum}[1]{``\nameref{#1}''~(\S\ref{#1})}

\newcommand{\pslabel}[1]{\phantomsection\label{#1}}
\newcommand{\subtab}[2][c]{\begin{tabular}{@{}#1@{}}#2\end{tabular}}

\newcommand{\refrem}[1]{Remark~\ref{#1}}
\newcommand{\refGad}[1]{Gadget~\ref{#1}}
\newcommand{\refgad}[1]{gadget~\ref{#1}}

\providecommand{\eqref}[1]{Equation~(\ref{#1})}
\newcommand{\eqcomment}{\mbox{// }}
\newcommand{\elcomment}[1]{\mbox{\quad// #1}}



%%%%%%%%%%%%%%%%%%%%%%%% Commands related to bibliography

%% The "aa" prefix in the commands denotes "Also at"


\newcommand{\iacrurl}[1]{\href{https://eprint.iacr.org/#1}{\nolinkurl{ia.cr/#1}}} %% ia.cr/\hspace{0em}#1
\newcommand{\aaiacr}[1]{Also at \iacrurl{#1}}
\newcommand{\eprintiacr}[1]{IACR Cryptology ePrint Archive, \iacrurl{#1}}  % when reference only exists at eprint
\newcommand{\seealsoiacr}[1]{See also \iacrurl{#1}}

\newcommand{\myhrefsplittwo}[3]{\href{#1}{#2}\hspace{0em}\href{#1}{#3}}  %% useful to enable a url to break line at a non usual place


\newcommand{\aaarxiv}[1]{Also at arXiv:\href{https://arxiv.org/abs/#1}{#1}}

%% \aaECCCTR{YYYY}{YY}{NNN}
\newcommand{\aaECCCTR}[3]{Also at ECCC~\href{https://eccc.weizmann.ac.il/report/#1/#3}{TR#2/#3}}
\newcommand{\seealsoweizmann}[2]{See also \href{https://eccc.weizmann.ac.il/report/#1/#2}{TR #1-#2}}

\newcommand{\doiurl}[1]{\href{https://doi.org/#1}{#1}}
% \printdoi: to use sometimes inside field addendum; should match the style used by biblatex in field doi
\newcommand{\printdoi}[1]{DOI:\doiurl{#1}}

\def\Proc{Proc.} % Abbreviation of ``Proceedings of the''

\DeclareListWrapperFormat{publisher}{Pub.\ by \mkbibemph{#1}}


%%%%%%%%%%%%%%%%%%%%%%%% Commands

\def\fillindesc{\red{$<$fill with description$>$}}
\newcommand{\emphbrkt}[1]{\ensuremath{\left\langle\emph{#1}\right\rangle}}
\newcommand{\mathfullstop}{\enspace.}
\def\ifempty#1{\def\temp{#1} \ifx\temp\empty }  % mathml doesn't support ifthenelse


%%%%%%%%%%%%%%%%%%%%%% Indexed environments

\newtheoremstyle{recbfstyle}{1em}{.5em}{\rm}{0em}{\bfseries}{. }{.5em}{\fontfamheads{\thmname{#1 }}{\thmnumber{#2}}\bfseries{}\ifthenelse{\equal{#3}{}}{}{ --- \thmnote{#3}}}
\theoremstyle{recbfstyle}
\newtheorem{definition}{Definition}[section]
\newtheorem{theorem}{Theorem}[section]
\newtheorem{remark}{Remark}[chapter]

%%% \newtheoremstyle{mystyle}%    % Name
%%%   {}%                         % Space above
%%%   {}%                         % Space below
%%%   {\itshape}%                 % Body font
%%%   {}%                         % Indent amount
%%%   {\bfseries}%                % Theorem head font
%%%   {.}%                        % Punctuation after theorem head
%%%   { }%                        % Space after theorem head, ' ', or \newline
%%%   {}%                         % Theorem head spec (can be left empty, meaning `normal')

%% style for recommendations and requirements

\theoremstyle{recbfstyle}
\newtheorem{recommendation}{Recommendation}[chapter]
\newtheorem{requirement}{Requirement}[chapter]


%%%%%%%%%%%%%%%%%%%%%%%%%%%%%%%%%%%%%%%%%%%%%%%%%%%%%%%%%%%%%%%%%%%%%%%%%%%%%%
%%%%%% Gadgets

\newcommand{\defgadget}[5]{{%   {name}{args}{ancilla}{constraints}{comment}
  \par
  \def\endofwitness{:}
  \let\oldelcomment=\elcomment
  \def\elcomment{\def\endofwitness{}\oldelcomment}
  \ensuremath{
  \begin{array}{l}
    \textsc{#1}(#2) := \{ 
    \ifempty{#5}\else{\quad \mbox{// {#5}} }\fi\\
    \mbox{}\quad\begin{array}{l}
       \ifempty{#3}\else#3\endofwitness \\ \fi
        #4 \; 
    \end{array} \\
  \}
  \end{array}
  }
}}

\newcommand{\usegadget}[2]{%   {name}{args}
  \ensuremath{ \textsc{#1}(#2) }
}

\newcommand{\namegadget}[1]{%   {name}
  \ensuremath{ \textsc{#1} }}


%%%%%%%%%%%%%%%%%%%%%%%%%%%%%%%%%%%%%%%%

\definecolor{NPcolor}{rgb}{0,0.3,0}
\newcommand{\NPstatement}[3]{{\color{NPcolor}\begin{itemize}[nosep]\item[] #1; \item[] #2; \item[] #3.\end{itemize}}}
\newcommand{\NPstatementNOperiod}[3]{{\color{NPcolor} \begin{itemize}[nosep] \item[] #1; \item[] #2; \item[] #3 \end{itemize}}}
\newcommand{\pkB}{\mathsf{pkB}}



%%%%%%%%%%%%%%%%%%%%%%%%%%%%%%%%%%%%%%%%%%%%%%%%%%%%%%%%%%%%%%%%%%%%%%%%%%%%%%
%%%%%% Template for proposed future figures
\newcounter{cntFF}[chapter]\setcounter{cntFF}{0}
\renewcommand{\thecntFF}{FF\thechapter.\arabic{cntFF}}
\def\futfigheadtext{Proposed future figure (illustration, diagram, ...):}
\def\futfigfoottext{}
%% The optional argument is a label that allows referencing this future figure
\newcommand{\futfig}[2][]{\begingroup\nolinenumbers\vskip.5em\setlength{\fboxsep}{.5em}\fbox{{\begin{minipage}{\textwidth-2\fboxsep-2\fboxrule}\centering\conditionalinternallinenumbers\setlinenumbermargin{1em}
    \textbf{\refstepcounter{cntFF}\ifthenelse{\equal{#1}{}}{}{\label{#1}}\textbf{\thecntFF.} \futfigheadtext}\vskip.5em
    #2
    \ifthenelse{\equal{\futfigfoottext}{}}{}{\vskip.5em\restoline{\futfigfoottext}}
    \end{minipage}}}\vskip.5em\endgroup}


%%%%%%%%%%%%%%%%%%%%%%%%%%%%%%%%%%%%%%%%%%%%%%%%%%%%%%%%%%%%%%%%%%%%%%%%%%%%%%
%%%%%% TO-DO's

\newcommand{\TODO}[1][]{\dtcolornote[TODO]{red}{#1}}

\definecolor{wantedcolor}{rgb}{0.5,0,0}
\newcommand{\WANTED}[1][elaboration]{\dtcolornote[CONTRIBUTION WANTED]{wantedcolor}{#1}}

\renewcommand{\WANTED}[1][]{\begingroup\nolinenumbers\vskip.5em\setlength{\fboxsep}{.75em}\fbox{{\begin{minipage}{\textwidth-2\fboxsep-2\fboxrule}\conditionalinternallinenumbers\setlinenumbermargin{1em}
    \color{wantedcolor}\textbf{Contribution wanted}\ifthenelse{\equal{#1}{}}{}{:~}#1
    \end{minipage}}}\vskip.75em\endgroup}


%\setlength{\marginparwidth}{2cm}  %%% temporary command to avoid warning when using comments
%\usepackage[colorinlistoftodos]{todonotes} 
%\newcommand{\scrtodo}[2]{\begingroup\setstretch{.5}\todo{\tiny #1}{#2}\endgroup}


%%%%%%%%%%%%%%%%%%%%%%%%%%%%%%%%%%%%%%%%%%%%%%%%%%%%%%%%%%%%%%%%%%%%%%%%%%%%%%
%%%%%%%%%%%%%%%%%%%%%%%% Command to create file based on expanding macros
% \filecontentsspecials|[]: https://tex.stackexchange.com/questions/534035/macros-commands-inside-a-filecontents-environment-does-not-expand
% needs to be run once before each \filecontents* that requires expanding some macros

\def\filecontentsspecials#1#2#3{
  \global\let\ltxspecials\dospecials
  \gdef\dospecials{\ltxspecials
    \catcode`#1=0
    \catcode`#2=1
    \catcode`#3=2
    \global\let\dospecials\ltxspecials
  }
}


%%%%%%%%%%%%%%%%%%%%%%%%%%%%%%%%%%%%%%%%%%%%%%%%%%%%%%%%%%%%%%%%%%%%%%%%%%%%%%
%%%%%%%%%%%%%%%%%%%%%%%% Symbols and abbreviations

\newcommand{\shortleftarrow}{\ensuremath{\scalebox{1}{\ensuremath{\leftarrow}}}}

\newcommand{\Comm}{\textmd{Comm}}  %%% \ensuremath{\mathcal{C}} %commitment

\newcommand{\prov}{\tops{\ensuremath{P}}{P}}
\newcommand{\veri}{\tops{\ensuremath{V}}{V}}

\newcommand{\XOR}{\texttt{XOR}}
\newcommand{\AND}{\texttt{AND}}
\newcommand{\NOT}{\texttt{NOT}}

\newcommand{\privsetP}{\tops{\ensuremath{\textmd{PrivateSetup}_{\prov}}}{PrivateSetup\_\prov}}
\newcommand{\privsetV}{\tops{\ensuremath{\textmd{PrivateSetup}_{\veri}}}{PrivateSetup\_\veri}}
\newcommand{\setP}{\tops{\ensuremath{\textmd{setup}_{\prov}}}{setup\_\prov}}
\newcommand{\setV}{\tops{\ensuremath{\textmd{setup}_{\veri}}}{setup\_\veri}}
\newcommand{\setR}{\tops{\ensuremath{\textmd{setup}_{R}}}{setup\_R}}
\newcommand{\params}{\textit{parameters}}  %LB: consider using params

\newcommand{\field}{\ensuremath{\mathbb{F}}}  %needs package amssymb

\newcommand{\shall}{{\ttfamily shall}}
\newcommand{\should}{{\ttfamily should}}
\newcommand{\can}{{\ttfamily can}}
\newcommand{\neednot}{{\ttfamily need not}}


%%%%%% Crypto primitives

\newcommand{\commit}{\ensuremath{\mathsf{Commit}}}
\newcommand{\open}{\ensuremath{\mathsf{Open}}}
\newcommand{\invalid}{\ensuremath{\mathsf{invalid}}}

\newcommand{\owf}{\ensuremath{\mathsf{owf}}}

%%%%%% VARIOUS ABBREVIATED COMMANDS
\def\s{\ensuremath{\sigma}}
\def\field{\ensuremath{\mathbb F}}
\def\F{\field}
\def\cF{\ensuremath{\cal F}}
\def\N{\ensuremath{\mathbb N}}
\def\bC{\ensuremath{\mathbf C}}
\def\params{\ensuremath{params}}


%%%%%% Constraint systems

\newcommand{\defem}[1]{\emph{\textbf{#1}}}
\newcommand{\enlargeparens}{{\rule{0pt}{2.2ex}}}
\newcommand{\ip}[2]{\left\langle #1 , #2 \right\rangle}
\newcommand{\bigip}[2]{\ip{\enlargeparens #1}{#2}}

\newcommand{\FieldSize}{\ensuremath{q}}
\newcommand{\FieldBits}{\ensuremath{n}}
\newcommand{\Field}{\ensuremath{\mathbb{F}_{\FieldSize}}} %_\FieldSize
\newcommand{\NumConstr}{\ensuremath{n_{\mathsf{c}}}}
\newcommand{\NumWit}{\ensuremath{{n_{\mathsf{w}}}}}
\newcommand{\NumInst}{\ensuremath{{n_{\mathsf{x}}}}}
\newcommand{\NumConc}{\ensuremath{{n_{\mathsf{z}}}}}
\newcommand{\inst}{\ensuremath{x}}
\newcommand{\wit}{\ensuremath{w}}
\newcommand{\conc}{\ensuremath{z}}
\newcommand{\vinst}{\ensuremath{\vec{\inst}}}
\newcommand{\vwit}{\ensuremath{\vec{\wit}}}
\newcommand{\vconc}{\ensuremath{\vec{z}}}
\newcommand{\lcoef}{\ensuremath{a}}
\newcommand{\rcoef}{\ensuremath{b}}
\newcommand{\ocoef}{\ensuremath{c}}
\newcommand{\System}{\ensuremath{\mathcal{S}}}
\newcommand{\Rel}{\ensuremath{\mathcal{R}}}
\newcommand{\Lang}{\ensuremath{\mathcal{L}}}

\newcommand{\zo}{\{0,1\}}
\newcommand{\NN}{\ensuremath{\mathbb{N}}}
\newcommand{\ZZ}{\ensuremath{\mathbb{Z}}}