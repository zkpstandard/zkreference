%%%%%%%%%%%%%%%%%%%%%%%%%%%%%%%%%%%%%%%%%%%%%%%%%%%%%%
%%%%%%%%%%%% Packages

\usepackage{iftex} %%%provides commands: \ifPDFTeX, \ifXeTeX, and \ifLuaTeX 

%%%%%%%%%%%%%%%%%%%%%%%%%%%%%%%%%%%%%%%%%%%%%%%%%%%%%%
%%%%%% Font related

\ifPDFTeX
	\usepackage[utf8]{inputenc}
	\usepackage{bold-extra}  % if using PdfTeX, allows bold small caps when compiling with PdfTex?, e.g., {\bfseries\textsc Changed:}
	\usepackage{cmap}
	\pdfsuppresswarningpagegroup=1 \pdfcompresslevel=9 \pdfminorversion=8 \pdfobjcompresslevel=9
\fi

\ifXeTeX\else %i.e., if PdfTex or LuaLaTeX
	\usepackage[T1]{fontenc} % does it ensure that undirected double-quotes remain undirected
	\usepackage{microtype}   % Produces warning, if using LuaLaTeX, about overwriting function `keepligature'
\fi

\usepackage{lmodern}

\usepackage{amsfonts} % allows \mathbb
\usepackage{amsmath} % Or mathtools, %for environment aligned

\usepackage{amsthm} %%% Packages theorem and amsthm clash in the definition of \theoremstyle
%\usepackage{theorem}  %allows simple definition of new theorem-like environments, e.g., 'remark'

\usepackage{amssymb} % allows \mathbb{F}
\usepackage{textcomp} %allows \textrightarrow (arrow in text mode)
\usepackage{bm} %command \bm allows bold math

%\usepackage{lmodern} %used by default?
%%%\usepackage{times} %newtxtext has font simultaneously bold and small caps
%\usepackage{newtxtext,newtxmath} %newtxtext,newtxmath  %package `newtxtext' is incompatible with package `textcomp'
\usepackage{anyfontsize} %allows command \fontsize{size}{skip}

%if using XeTeX or LuaLaTeX, the compilation might be unable to use font simultaneously bold and small caps (see command \Chan)

\usepackage[normalem]{ulem} %allows command \sout for strikethrough text
			%option normalem makes the \emph{} command remain untouched. Needed to get correctly formatted refs

\usepackage{textgreek} % allows \textSigma

\ifPDFTeX\else\usepackage{polyglossia}\setmainlanguage{english}\fi

%%%%%%%%%%%%%%%%%%%%%%%%%%%%%%%%%%%%%%%%%%%%%%%%%%%%%%
%%%%%% Page and Geometry related

%%% \usepackage{geometry} % may be needed to call \newgeometry later on
\usepackage[margin=1in]{geometry}  %%% \usepackage[left=1in,right=1in,top=1in,bottom=1in]{geometry}

%%% The package typearea will redefine the following lengths:
\newcommand{\newsetlength}[2]{\newlength{#1}\setlength{#1}{#2}}
\newsetlength{\oldtopmargin}{\topmargin}
\newsetlength{\oldheadheight}{\headheight}
\newsetlength{\oldoddsidemargin}{\oddsidemargin}
\newsetlength{\oldevensidemargin}{\evensidemargin}
\newsetlength{\oldtextwidth}{\textwidth}
\newsetlength{\oldtextheight}{\textheight}


\usepackage{silence} %useful to silence some harmless/unavoidab warnings 
%\WarningFilter{typearea}{Bad type area settings!}
%\WarningFilter*{geometry}{Over-specification in }

%\usepackage[usegeometry]{typearea} % needed to change to landscape in the middle of the document
%option `usegeometry' enables the correct resizing of text area when adding pages in landscape mode


\setlength{\topmargin}{\oldtopmargin}
\setlength{\headheight}{\oldheadheight}
\setlength{\oddsidemargin}{\oldoddsidemargin}
\setlength{\evensidemargin}{\oldevensidemargin}
\setlength{\textwidth}{\oldtextwidth}
\setlength{\textheight}{\oldtextheight}


\usepackage{setspace} %\setstretch{1}, %\setstretch{1.25}

\usepackage{pdflscape} %alllows landscape environment with automatic rotation of the page %for use when integrating with other documents
\usepackage{afterpage}

\usepackage{fancyhdr}\renewcommand{\headrulewidth}{0pt}

%https://tex.stackexchange.com/questions/59785/how-to-execute-command-on-every-table-column
%\usepackage{collcell} %command \collectcell\function and \endcollectcell to apply a function to a cell



%%%%%%%%%%%%%%%%%%%%%%%%%%%%%%%%%%%%%%%%%%%%%%%%%%%%%%5
%%%%%% Diverse packages

\usepackage{comment}

%%% biblatex: Some options need to be called when loading. Others can be called later with \ExecuteBibliographyOptions
\usepackage[backend=biber,style=alphabetic,citestyle=alphabetic]{biblatex}

\usepackage{twoopt} %allows \newcommandtwoopt to define commands with two optional arguments
%\newcommandtwoopt{cmdname}[num][default1][default2]{actions}

\usepackage{multicol}

%%%https://tex.stackexchange.com/questions/69595/marginnote-always-on-right-side-of-the-page
%% make the margin always appear on the right side if using the twocol option of the documentclass `article'
%% \makeatletter
%% \patchcmd{\@addmarginpar}{\ifodd\c@page}{\ifodd\c@page\@tempcnta\m@ne}{}{}
%% \makeatother
%% \reversemarginpar

\usepackage[sf,bf]{titlesec}  % the sf option makes all headings be with sans serif

\usepackage{enumitem} %allows command \setlist, e.g., \setlist[itemize]{leftmargin=1em}



\usepackage{graphicx,color}
\usepackage{xcolor}
 \definecolor{urldocblue}{RGB}{17,85,204}
 \definecolor{ongoingred}{RGB}{224,102,102}
 \definecolor{darkblue}{RGB}{0,0,191}

%%%%%%%%%%%%%%%%%%%%%%%%%%%%%%%%%%%%%%%%
%%%%%% Related to Tables

\usepackage{caption}
\captionsetup[table]{skip=10pt,labelfont=bf}

\usepackage{diagbox} %allows command \diagbox to indicate the label for rows and columns in a single cell

%% Option longtable when loading package lineno leads to LaTeX Warning: Command \LT@start has changed. So I'll simply call longtable later
\usepackage[edtable,mathlines]{lineno} % LB: the command \linenumbers will be conditionally called depending on boolShowLineNumbers, because \thelinenumber will be conditionally changed across the document, when inside tabular and minipage environments
\usepackage{edtable}
\usepackage{longtable} %environment longtable replaces tabular, and no need to wrap it within table
%allows table to break across pages; allows command \endhead to define what rows to repeat

\usepackage{colortbl} %allows coloring of rows (using \rowcolor) and columns in tabular
\definecolor{colorRowHead}{rgb}{0.95,0.95,0.95}

\usepackage{array} % for extended column definitions  		  % allows using >{...} and <{...} in specifying columns

\usepackage{booktabs,multirow}
\usepackage{float} %allows option H for floats, e.g., \begin{table}[H], which is stronger than \begin{table}{!h}

%%%%%%%%%%%%%%%%%%%%%%%%%%%%%%%%%%%%%%%%
%%%%%% Related to comments, counters and hyper-references

\usepackage{pdfcomment}  % clashes with dtrt, so removed the latter
% pdfcomment: when compiling a pdfcomment using xetex or lualatex, the opening of the PDF file in a fresh instance of the reader complains about a missing font: "Cannot extract the embedded font `......+LMSans10--Bold-Identity-H'. Some characters may not display or print correctly"
\usepackage{attachfile2}

\PassOptionsToPackage{hyphens}{url} %% allow urls to break the line
\usepackage{cleveref}
\usepackage[all]{hypcap}

\usepackage{assoccnt}
\newcounter{mysec}[chapter]
\DeclareAssociatedCounters{section}{mysec}

\usepackage{tocloft} %command \setcounter{tocdepth}{1}, also allows creating other lists
\usepackage{shorttoc}

\usepackage{bookmark}\bookmarksetup{startatroot} 
\bookmarksetup{open,addtohook={\ifnum\bookmarkget{level}=0 \bookmarksetup{bold}\fi}}


%%%%%%%%%%%%%%%%%%%%%%%%%%%%%%%%%%%%%%%%
%%%%%% Create diagrams

\usepackage{tikz}
\usetikzlibrary{mindmap,backgrounds} %https://www.overleaf.com/learn/latex/LaTeX_Graphics_using_TikZ:_A_Tutorial_for_Beginners_(Part_5)%E2%80%94Creating_Mind_Maps
