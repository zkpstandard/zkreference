%%%%%%%%%%%%%%%%%%%%%%%%%%%%%%%%%%%%%%%%%%%%%%%%%%%%%%%%%%%%
%%%%%%%%%%%%%%%%%%%%%%%%%%%%%%%%%%%%%%%%%%%%%%%%%%%%%%%%%%%%
\section{Backends: Cryptographic System Implementations}
\label{implem:backends}

The backend of a ZK proof implementation is the portion of the software that contains an implementation of the low-level cryptographic protocol. It proves statements where the instance and witness are expressed as variable assignments, and relations are expressed via low-level languages (such as arithmetic circuits, Boolean circuits, R1CS/QAP constraint systems or arithmetic constraint satisfaction problems).

The backend typically consists of a concrete implementation of the ZK proof system(s) given as pseudocode in a corresponding publication (see the \hyperref[chap:security]{Security Track} document for extensive discussion of these), along with supporting code for the requisite arithmetic operations, serialization formats, tests, benchmarking etc.

There are numerous such backends, including implementations of many of the schemes discussed in the \hyperref[chap:security]{Security Track}. 
Most have originated as academic research prototypes, and are available as open-source projects. 
Since the offerings and features of backends evolve rapidly, we refer the reader to the curated taxonomy at https://zkp.science for the latest information.

Considerations for the choice of backends include:

\begin{itemize}
\item ZK proof system(s) implemented by the backend, and their associated security, assumptions and asymptotic performance (as discussed in the Security Track document)
\item Concrete performance (see Benchmarks section)
\item Programming language and API style (this consideration may be satisfied by adherence to prospective ZK proof standards; see the the API and File Formats section)
\item Platform support
\item Availability as open source
\item Active community of maintainers and users
\item Correctness and robustness of the implementation (as determined, e.g., by auditing and formal verification)
\item Applications (as evidence of usability and scrutiny).
\end{itemize}

