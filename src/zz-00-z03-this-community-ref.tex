\presection{About this community reference}
\label{sec:prelim:about-this-community-reference}

	\revblock[rev:prelim:explain-context-of-community-reference]{\ref{it:editorial:about-this-community-reference}}
	This ``ZKProof Community Reference'' arises within the scope of the ZKProof open initiative, which seeks to mainstream zero-knowledge proof (ZKP) cryptography.
	This is an inclusive community-driven process that focuses on interoperability and security, aiming to advance trusted specifications for the implementation of ZKP schemes and protocols.


	ZKProof holds annual workshops, attended by world-renowned cryptographers, practitioners and industry leaders.
	These events are a forum for discussing new proposals, reviewing cutting edge projects, and advancing reference material.
	That is the genesis of this document, which intends to be a community-built reference for understanding and aiding the development of ZKP systems.


	The following items provide guidance for the expected development process of this document, which is open to contributions from and for the community.



\textbf{Purpose.}
	The purpose of developing the ZKProof Community Reference document is to provide, 
within the principles laid out by the \hyperref[sec:prelim:charter]{ZKProof charter}, 
a reference for the development of zero-knowledge-proof technology that is secure, practical and interoperable.


\textbf{Aims.}
	The aim of the document is to consolidate reference material developed and/or discussed in collaborative processes during the ZKProof workshops. 
	The document intends to be accessible to a large audience, including the general public, the media, the industry, developers and cryptographers.


\textbf{Scope.}
	The document intends to cover material relevant for its purpose --- the development of secure, practical and interoperable technology.
	The document can also elaborate on introductory concepts or works, to enable an easier understanding of more advanced techniques. 
	When a focus is chosen from several alternative options, the document should include a rationale describing comparative advantages, disadvantages and applicability. 
	However, the document does not intend to be a thorough survey about ZKPs, and does not need to cover every conceivable scenario.


\textbf{Format.}
	To achieve its accessibility goal, and considering its wide scope, the document favors the inclusion of: 
	a well defined structure (e.g., chapters, sections, subsections);
	introductory descriptions (e.g., an executive summary and one introduction per chapter); 
	illustrative examples covering the main concepts; 
	enumerated recommendations and requirements; 
	summarizing tables; 
	glossary of technical terms; 
	appropriate references for presented claims and results.


\textbf{Editorial methodology.}
	The development process of this community reference is proposed to happen in cycles of four phases:
	\begin{enumerate}[label=(\roman*),itemsep=0ex]
	\item \textbf{open discussion} during ZKProof workshops, with corresponding annotations to serve as reference for subsequent development;
	\item \textbf{content development}, by voluntary \emph{contributors}, according to a set of contribution proposals and during a defined period;
	\item \textbf{content integration} in the document, by the \emph{editors}, of submitted contributions; 
	\item \textbf{public feedback} on the state of development of the document, to be used as a basis of development in subsequent cycles of the process.
	\end{enumerate}
	The team of editors coordinates the process, 
promoting transparency by means of public calls for contributions and feedback, 
using editorial discretion towards the improvement of the document quality,
and enabling an easy way to identify the changes and their rationale.
