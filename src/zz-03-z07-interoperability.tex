%%%%%%%%%%%%%%%%%%%%%%%%%%%%%%%%%%%%%%%%%%%%%%%%%%%%%%%%%%%%
%%%%%%%%%%%%%%%%%%%%%%%%%%%%%%%%%%%%%%%%%%%%%%%%%%%%%%%%%%%%
\section{Extended Constraint-System Interoperability}
\label{implem:interoperability}

The following are stronger forms of interoperability which have been identified as desirable by practitioners, and are to be addressed by the ongoing standardization effort.


%%%%%%%%%%%%%%%%%%%%%%%%%%%%%%
\subsection{Statement and witness formats}
In the R1CS File Format section and associated resources, we define a file format for R1CS constraint systems. There remains to finalize this specification, including instances and witnesses. This will enable users to have their choice of frameworks (frontends and backends) and streaming for storage and communication, and facilitate creation of benchmark test cases that could be executed by any backend accepting these formats.
 
Crucially, analogous formats are desired for constraint system languages other than R1CS.


%%%%%%%%%%%%%%%%%%%%%%%%%%%%%%
\subsection{Statement semantics, variable representation \& mapping}

Beyond the above, there’s a need for different implementations to coordinate the semantics of the statement (instance) representation of constraint systems. For example, a high-level protocol may have an RSA signature as part of the statement, leaving ambiguity on how big integers modulo a constant are represented as a sequence of variables over a smaller field, and at what indices these variables are placed in the actual R1CS instance.

Precise specification of statement semantics, in terms of higher-level abstraction, is needed for interoperability of constraint systems that are invoked by several different implementations of the instance reduction (from high-level statement to the actual input required by the ZKP prover and verifier). One may go further and try to reuse the actual implementation of the instance reduction, taking a high-level and possibly domain-specific representation of values (e.g., big  integers) and converting it into low-level variables. This raises questions of language and platform incompatibility, as well as proper modularization and packaging.

Note that correct statement semantics is crucial for security. Two implementations that use the same high-level protocol, same constraint system and compatible backends may still fail to correctly interoperate if their instance reductions are incompatible -- both in completeness (proofs don’t verify) or soundness (causing false but convincing proofs, implying a security vulnerability). Moreover, semantics are a requisite for verification and helpful for debugging.

Some backends can exploit uniformity or regularity in the constraint system (e.g., repeating patterns or algebraic structure), and could thus take advantage of formats and semantics that convey the requisite information.

At the typical complexity level of today’s constraint systems, it is often acceptable to handle all of the above manually, by fresh re-implementation based on informal specifications and inspection of prior implementation. We expect this to become less tenable and more error prone as application complexity grows.

%%%%%%%%%%%%%%%%%%%%%%%%%%%%%%
\subsection{Witness reduction}

Similar considerations arise for the witness reduction, converting a high-level witness representation (for a given statement) into the assignment to witness variables. For example, a high-level protocol may use Merkle trees of particular depth with a particular hash function, and a high-level instance may include a Merkle authentication path. The witness reduction would need to convert these into witness variables, that contain all of the Merkle authentication path data (encoded by some particular convention into field elements and assigned in some particular order) and moreover the numerous additional witness variables that occur in the constraints that evaluate the hash function, ensure consistency and Booleanity, etc.

The witness reduction is highly dependent on the particular implementation of the constraint system. Possible approaches to interoperability are, as above: formal specifications, code reuse and manual ad hoc compatibility.

%%%%%%%%%%%%%%%%%%%%%%%%%%%%%%
\subsection{Gadgets interoperability}
At a finer grain than monolithic constraint systems and their assignments, there is need for sharing subcircuits and gadgets. For example, libsnark offers a rich library of highly optimized R1CS gadgets, which developers of several front-end compilers would like to reuse in the context of their own constraint-system construction framework.

While porting chunks of constraints across frameworks is relatively straightforward, there are challenges in coordinating the semantics of the externally-visible variables of the gadget, analogous to but more difficult than those mentioned above for full constraint systems: there is a need to coordinate or reuse the semantics of a gadget’s externally-visible variables, as well as to coordinate or reuse the witness reduction function of imported gadgets in order to converts a witness into an assignment to the internal variables.

As for instance semantics, well-defined gadget semantics is crucial for soundness, completeness and verification, and is helpful for debugging.

%%%%%%%%%%%%%%%%%%%%%%%%%%%%%%
\subsection{Procedural interoperability}
An attractive approach to the aforementioned needs for instance and witness reductions (both at the level of whole constraint systems and at the gadget level) is to enable one implementation to invoke the instance/witness reductions of another, even across frameworks and programming languages.

This requires communication not of mere data, but invocation of procedural code. Suggested approaches to this include linking against executable code (e.g., .so files or .dll), using some elegant and portable high-level language with its associated portable, or using a low-level portable executable format such as WebAssembly. All of these require suitable calling conventions (e.g., how are field elements represented?), usage guidelines and examples.

Beyond interoperability, some low-level building blocks (e.g., finite field and elliptic curve arithmetic) are needed by many or all implementations, and suitable libraries can be reused. To a large extent this is already happening, using the standard practices for code reuse using native libraries. Such reused libraries may offer a convenient common ground for consistent calling conventions as well.

%%%%%%%%%%%%%%%%%%%%%%%%%%%%%%
\subsection{Proof interoperability}
Another desired goal is interoperability between provers and verifiers that come from different implementations, i.e., being able to independently write verifiers that make consistent decisions and being able to re-implement provers while still producing proofs that convince the old verifier.

This is especially pertinent in applications where proofs are posted publicly, such as in the context of blockchains (see the Applications Track document), and multiple independent implementations are desired for both provers and verifiers.

To achieve such interoperability between provers and verifiers, they must agree on all of the following:
\begin{itemize}
    \item ZK proof system (including fixing all degrees of freedom, such as choice of finite fields and elliptic curves)
    \item Instance and witness formats (see above subsection)
    \item Prover parameters formats
    \item Verifier parameters formats
    \item Proof formats
    \item A precise specification of the constraint system (e.g., R1CS) and corresponding instance and witness reductions (see above subsection).
\end{itemize}

Alternatively: a precise high-level specification along with a precisely-specified, deterministic frontend compilation.


%%%%%%%%%%%%%%%%%%%%%%%%%%%%%%
\subsection{Common reference strings}
There is also a need for standardization regarding Common Reference String (CRS), i.e., prover parameters and verifier parameters. First, interoperability is needed for streaming formats (communication and storage), and would allow application developers to easily switch between different implementations, with different security and performance properties, to suit their need. Moreover, for Structured Reference Strings (SRS), there are nontrivial semantics that depend on the ZK proof system and its concrete realization by backends, as well as potential for partial reuse of SRS across different circuits in some schemes (e.g., the Powers of Tau protocol). 

