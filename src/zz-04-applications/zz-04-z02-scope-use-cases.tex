\section{Types of verifiability}   %Verifiability scope of use-cases
\label{apps:scope-use-cases}


\paragraph{Verifiability type.}

When designing ZK based applications, one needs to keep in mind which of the following three models (that define the functionality of the ZKP) is needed:

\begin{enumerate}

    \item \textbf{Public.}
		Publicly verifiable as a requirement: a scheme / use-case where 
	there is a system requirement that the proofs are transferable. %%%, where such property is actually a requirement of the system. 
		

		
    \item \textbf{Designated.} 
		Designated verifier as a security feature: only the intended receiver of the proof can verify it, making the proof non-transferable. 
		This property can apply to both interactive and non-interactive ZKPs.

		
    \item \textbf{Optional.} %The final model is one where neither of the above is needed: a ZK where 
		There is no need to be able to transfer but also no non-transferability requirement. %Again, 
		This property is applicable both in the interactive and in the non-interactive model.

\end{enumerate}


	\refsec{sec:paradigms:interactivity:transferability-deniability} discusses transferability vs.\ deniability, %%%public verifiability vs. designated verifiability

which is strongly related to aspects of public verifiability vs.\ designated verifiability,
both in the interactive and in the non-interactive settings.
	As a use-case example, consider some application related to blockchain currency, where aspects of user-privacy and regulatory-control are relevant.
	
	
	Publicly-verifiable ZKPs can be appropriate when the validity of a transaction should be public 
(e.g., so that everyone knows that some asset changed owner), while some supporting data needs to remain private 
(e.g., the secret key of a blockchain address, controlling the ownership of the asset).
	However, sometimes even the statement being proven should remain private beyond the scope of the verifier, and therefore a non-transferable proof	should be used.
	This may apply for example to a proof of having enough funds available for a purchase, or also of knowing the secret key of a certain blockchain address.
	Alice wants to prevent Bob from using the received proof to convince Charley of the claims made by Alice.
	For that purpose, Alice can perform a deniability interactive proof with Bob.
	Alternatively, Alice can send to Bob a (non-interactive) proof transcript built for Bob as a \emph{designated verifier}.
	Depending on the use case, both public-verifiability and designated-verifiability may make sense as an application goal, and it is important to distinguish between both.
%%%For example, using a designated-verifier proof of a secret-key, related to a public key, Alice can convince Bob, without the latter being able to transfer the proof to any other external party. 
%%%“Compliance for regulation” can make sense in the second model (albeit depending on the 1940 use-case).
%%%In general, the credential system can be in both of the last two models, given the extra constraints that would make it belong to the second model.


%%%%%%%%%%%%%%%%%%%%%%%%%%%%%%%%%%%%%%%%%%%%%%%%%%%%%%%%%%%%%%%%%%%%%%%
%%%\paragraph{Designated-verifier proofs.}

	
	The ``designation of verifiers'' allows resolving possible conflicts between authenticity and privacy \cite{1996:eurocrypt:designated-verifier-proofs}.
	For example, a voting center wants only Bob to be convinced that the vote he cast was counted;
	the voting center designates Bob to be the one convinced by the validity of the proof, 
in order to prevent a malicious coercer to force him to prove how he voted.
	Since the designated-verifier proofs are non-transferable, Bob cannot transfer the proof even if he wants to.


	Suppose Alice wants to convince \emph{only} Bob that a statement $\theta$ is true.
	For that purpose, Alice can prove the disjunction ``Either $\theta$ is true or I know the secret key of Bob''.
	Given that Bob knows his own secret key, Bob could have produced such proof by himself.
	Therefore, a third party Charlie will not be convinced that $\theta$ is true after seeing such proof transcript sent from Bob.
	This holds even if Bob shares his secret key to Charlie, or if the key has been publicly leaked.

	%%%For example, \cite{1996:eurocrypt:designated-verifier-proofs} describes one protocol based on trapdoor commitments.
	Designated proofs are possible both in the interactive and non-interactive settings.
	In the interactive setting (e.g., proving being the signer of an undeniable signature) the prover has the ability to control when the verification takes place.
	However, in general (without a designated-verifier approach) the prover may be unable to control who 
is able to verify the proof, namely if the verifier is acting as a relay to another controlling party.
	The use of a designated proof has the potential to solve this problem.

%	Non-interactivity can be based on the Fiat-Shamir technique \cite{1987:crypto-86:How-to-Prove-Yourself},
%where the challenge of the interactive protocol is instead computed as 
%the hash of previous elements in the interactive protocol.
	

