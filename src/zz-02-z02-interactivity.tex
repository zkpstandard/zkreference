%%%%%%%%%%%%%%%%%%%%%%%%%%%%%%%%%%%%%%%%%%%%%%%%%%%%%%%%%%%%
%%%%%%%%%%%%%%%%%%%%%%%%%%%%%%%%%%%%%%%%%%%%%%%%%%%%%%%%%%%%
\section{Interactivity}
\label{paradigms:interactivity} 

% Include here a detailed discussion of interactive vs. non-interactive protocols
  % advantages and disadvantages, security properties, implementation concerns, ...

\revblock[rev:interactivity-paradigm]{\ref{it:interactivity:intro}}


Several of the proof systems described in the Taxonomy of Constructions given in Section \ref{paradigms:taxonomy}
are interactive, including classical interactive proofs (IPs), IOPs, and linear IOPs. 
This means that the verifier sends multiple challenge messages to the prover,
with the prover replying to challenge $i$ before receiving challenge $i+1$;
soundness relies on the prover being unable to predict challenge $i+1$ when it responds to challenge $i$.
The other proof systems from the Taxonomy of Constructions are non-interactive, namely classical PCPs and linear PCPs. 
All of these proof systems can be combined with cryptographic
compilers to yield argument systems that may or may not be interactive, depending on the compiler.



\subsection{Advantages of Interactive Proof and Argument Systems}
\label{advantagesofinteraction}
\begin{enumerate}[label=\alph*.]
\item \underline{Efficiency and Simplicity}. Interactive proof systems can be simpler or more efficient than non-interactive ones.
As an example, researchers introduced the IOP model \cite{2016:tcc:IOPs, 2016:stoc:Constant-round-Interactive-Proofs-for-Delegating-Computation}, which is interactive, in part because interactivity allowed for circumventing efficiency bottlenecks arising
in state of the art PCP constructions \cite{2013:STOC:concrete-efficiency-PCPs}. 
As another example, some argument systems derived from IPs \cite{2018:SP:Doubly-efficient-zkSNARKs-without-trusted-setup, 2019:crypto:libra}
have substantially better space complexity for the prover (a key scalability bottleneck) than state of the art PCPs \cite{2013:STOC:concrete-efficiency-PCPs} or linear PCPs \cite{2013:QSPs-and-succinct-NIZKs-without-PCPs, 2016:Eurocrypt:On-the-Size-of-Pairing-Based-Non-interactive-Arguments}. 

Yet, if an interactive protocol is public coin, it can be rendered non-interactive and publicly verifiable in most settings via the Fiat-Shamir transformation (see \refsec{paradigms:taxonomy:compilers-crypto}), often with little loss in efficiency. This means that protocol designers have the freedom to leverage interactivity as a ``resource'' to simplify protocol design, improve efficiency, weaken or remove trusted setup, etc., and still have the option of obtaining a non-interactive argument using the Fiat-Shamir transformation.

(Applying the Fiat-Shamir heuristic to an interactive protocol to obtain a non-interactive argument may increase soundness error, and may transform statistical security to computational security (see Section {\color{red} insert reference to section(s) containing discussion of computational and statistical security, e.g., new section 3.5.4}). However, recent works \cite{2016:tcc:IOPs, 2019:STC:Fiat-Shamir-from-practice-to-theory} show 
that when the transformation is applied to specific IP, IOP, and linear IOP protocols of both practical and theoretical interest, the blowup in soundness error is only polynomial in the number of rounds of interaction). 


\item \underline{Setup.} Cryptographic compilers for linear PCPs currently require a structured reference string (SRS) (see \refsec{implem:correctness:SRS-gen}). 
Here, an SRS is a structured string that must be generated by a trusted third party during a setup phase, and soundness requires that any
trapdoor used during this trusted setup must not be revealed. In contrast, some compilers that apply to IPs, IOPs (as well as PCPs), and linear IPs
yields arguments in which the prover and the verifier need only access a uniform random string (URS), which can be obtained
from a common source of randomness. Such a setup is referred as \emph{transparent} setup in the literature.

\item \underline{Cryptographic Primitives.} Argument systems derived from IPs, IOPs, or linear IOPs also sometimes rely on more desirable cryptographic primitives. For example, IPs themselves are information-theoretically secure, relying on no cryptographic assumptions at all. And
in contrast to arguments derived from linear PCPs, those derived from IOPs rely only on symmetric-key cryptographic primitives (see, e.g., \cite{2016:tcc:IOPs}).
Finally, it has long been known how to obtain succinct \emph{interactive} arguments in the plain model based on falsifiable assumptions like collision-resistant
hash families \cite{1995:crypto:Improved-Efficient-Arguments}, but this is not the case for succinct \emph{non-interactive} arguments.



\item \underline{Non-transferability.} In some applications, it is essential that proofs be deniable or \emph{non-transferrable} (i.e., it must be impossible for a verifier to convince a third party of the validity of the statement; see Section {\color{red}  insert reference to ``Transferability vs. deniability'' section proposed to be inserted by NIST-PEC}). While these properties are not unique to interactive protocols, interaction offers a natural way to make proofs non-transferable (for details, see  Section \ref{section:transferability-deniability-interactivity}). 


\item  \underline{Interactivity May Limit Adversaries' Abilities.}  Interactive protocols can potentially be run with fewer bits of security and hence be more efficient. For example, interactive settings may allow
for the enforcement of a time limit for the protocol to terminate, limiting the runtime of attackers. Alternatively, 
in an interactive setting
it may be possible to ensure that adversaries only have one attempt to attack a protocol, 
while this will not be possible in many non-interactive settings. See Section {\color{red} insert reference to paragraph on ``An exception allowing lower computational security parameter.'' in NIST-PEC proposed changes} for details.

\item \underline{Interactivity May Be Inherent to Applications}. Many applications are inherently interactive. For example, real-world networking protocols involve multiple messages just to initiate a connection. In addition,
zero-knowledge protocols are often combined with other cryptographic primitives in applications (e.g., oblivious transfer). If the other primitives are interactive, then the final cryptographic protocol will be interactive regardless of whether the zero-knowledge protocol is non-interactive.
If an application is inherently interactive, it may be reasonable to leverage the interaction as a resource if it can render a protocol simpler, more efficient, etc. 


\end{enumerate}
\subsection{Disadvantages of Interactive Proof and Argument Systems}
\label{disadvantagesofinteraction}
\begin{enumerate}

\item \underline{Interactive protocols must occur online.} In an interactive protocol, the proof cannot simply be published or posted and checked later at the verifier's convenience, as can be done with non-interactive protocols.

\item \underline{Public Verifiability}. Many applications require that proofs be verifiable by any party at any time. 
Public verifiability
may be difficult to achieve for interactive protocols. This is because soundness of interactive 
protocols relies on the prover being unable to predict the next challenge it will receive in the protocol. Unless there is a publicly trusted source of unpredictable
randomness (e.g., a randomness beacon) and a way for provers to timestamp messages, it is not clear how any party other than
the one sending the challenges can be convinced that the challenges were properly generated, and the prover replied to challenge $i$ before learning challenge $i+1$.
See Section \ref{section:transferability-deniability-interactivity} below for further details.

\item \underline{Network latency can make interactive protocols slow.} If an interactive protocol consists of many messages sent over a network,
network latency may contribute significantly to the total execution time of the protocol.

\item \underline{Timing or Side Channel Attacks.} Because interactive protocols require the prover to send multiple messages, there may be more vulnerability to side channel or timing attacks compared to non-interactive protocols. Timing attacks will only affect zero-knowledge, not soundness, for public-coin protocols, because the verifier's messages are simply random coins, and timing attacks should not leak information to the prover in this case. In private coin protocols, both zero-knowledge and soundness may be affected by these attacks.

\item \underline{Concurrent Security.}  If an interactive protocol is not used in isolation, but is instead used in an environment where multiple interactive
protocols may be executed concurrently, then considerable care should be taken to ensure that the protocol remains secure. 
 See for example \cite[Section 2.1]{2013:CISS:a-short-tutorial-on-zero-knowledge}
and the references therein. Issues of concurrent execution security
are greatly mitigated for non-interactive protocols \cite{2006:eurocrypt:perfect-NIZK-for-NP}. 

\item \underline{Proof Length.} Currently, the zero-knowledge protocols with the shortest known proofs are based on linear PCPs, which are non-interactive. These proofs are just a few group elements (see Table \ref{tab:different-types-of-PCP}). While (public-coin) zero-knowledge protocols based on IPs or IOPs can be rendered non-interactive with the Fiat-Shamir heuristic, they currently produce longer proofs. The longer proofs may render these protocols unsuitable for some applications (e.g., public blockchain), but they may still be suitable for other applications (even related ones, like enterprise blockchain applications).


\end{enumerate}

\subsection{Nuances on transferability vs. interactivity}
\label{section:transferability-deniability-interactivity}
{\color{red} Insert NIST-PEC proposed paragraph on Nuances on transferability vs. interactivity.}\revblock[rev:deniability-transferability]{\ref{it:transferability-with-auth-channels}, \ref{it:deniability}}
