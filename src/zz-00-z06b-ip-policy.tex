\presection[Intellectual property]{Intellectual property --- expectations on disclosure and licensing}
\label{sec:prelim:IP}

\vspace{-1em}
\revblock[rev:intellectual-property]{\ref{it:intellectual-property}}
	ZKProof is an open initiative that seeks to promote the secure and interoperable use of zero-knowledge proofs. 
	To foster open development and wide adoption, it is valuable to promote technologies with open-source implementations, unencumbered by royalty-bearing patents.
	However, some useful technologies may fall within the scope of patent claims. 
	Since ZKProof seeks to represent the technology, research and community in an inclusive manner, it is  
valuable to set expectations about the disclosure of intellectual property and the handling of patent claims.


	The members of the ZKProof community are hereby strongly encouraged to provide information on known patent claims 
(their own and those from others)\revblock[rev:intellectual-property:clarify-whose-claims-are-in-scope]{\ref{it:intellectual-property:minor-tweak}}
potentially applicable to the guidance, requirements, recommendations, proposals and examples provided in ZKProof documentation, including by disclosing known pending patent applications or any relevant unexpired patent.
	Particularly, such disclosure is promptly required from the patent holders, or those acting on their behalf, as a condition for providing content contributions to the ``Community Reference'' and to ``Proposals'' submitted to ZKProof for consideration by the community.
	The ZKProof documentation will be updated based on received disclosures about pertinent patent claims.
	Please email information to \href{mailto:editors@zkproof.org}{editors@zkproof.org}.


	ZKProof aims to produce documents that are open for all and free to use.\revblock[rev:intellectual-property:extend-cc-license-expectation]{\ref{it:intellectual-property:cc-license-expectation}}
	As such, the content produced for publication within the context of the ZKProof Standards effort should be made available under a Creative Commons Attribution 4.0 International license.
	Furthermore, any technology that is promoted in said ZKProof documentation and that falls within patent claims should be made available under licensing terms that are reasonable, and demonstrably free of unfair discrimination, preferably allowing free open-source implementations.