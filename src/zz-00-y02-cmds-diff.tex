%%% This file: defines counters and commands useful for cross-referencing feedback notes and corresponding changes


%%%%%%%%%%%%%%%%%%%%%%%%%%%% Commands
%%% Define here new personalized commands for leaving comments

\newcommand{\popupColor}[3][blue]{\pdfcomment[color=#1,opacity=0.5,author={#2}]{#3}}
\newcommand{\suggest}[2][]{\popupColor[blue]{#1}{Suggestion: #2}}
\newcommand{\edited}[2][]{\popupColor[green]{#1}{Edited: #2}}
\newcommand{\smttodo}[2][]{\popupColor[red]{#1}{To-do: #2}}

\newcommand{\luis}[1]{\popupColor[blue]{Lu{\'i}s B.}{#1}}
\let\luiscom\luis
\newcommand{\luissug}[1]{\suggest[Lu{\'i}s B.]{#1}}
\newcommand{\luised}[1]{\edited[Lu{\'i}s B.]{#1}}
\newcommand{\luistodo}[1]{\smttodo[Lu{\'i}s B.]{#1}}
\newcommand{\luisin}[1]{\red{[[Lu{\'i}s B.: #1]]}}


%%%%% Mark revision blocks
\newcounter{cntRev}\renewcommand{\thecntRev}{E\arabic{cntRev}}


\newcommandtwoopt{\revblock}[3][][0pt]{\marginnote{\hspace*{#2}\resizebox{.85in}{!}{{\color[rgb]{1,0,0}%
			\begin{minipage}{1.1in}\vspace{0em}\refstepcounter{cntRev}%
				\ifNotEmpty{#1}{\label{#1}} %adds a labels to the review edit
				\thecntRev\ifNotEmptyPrepend{#3}{: }
			\end{minipage}}}}}
\newcommandtwoopt{\revblockmini}[3][][]{{\scriptsize\revblock[#1][#2]{#3}}}
%\renewcommand{\revblock}[2][]{}   % Uncomment this line to get ride of all margin notes


%\newcommand{\revblockRight}[3][]{\revblock[#1][#3]{#2}}  %%% to delete in favor of using \revblock with the second optional argument


\providecommand{\blue}[1]{{\color[rgb]{0,0,1}#1}}
\providecommand{\red}[1]{{\color[rgb]{1,0,0}#1}}
\providecommand{\dgreen}[1]{{\color[rgb]{0,.33,0}#1}}
\definecolor{mycoral}{rgb}{.9,.333,.2000}
\definecolor{brown}{rgb}{0.6,0.3,0}
\definecolor{LLgray}{rgb}{0.9,0.9,0.9}
\definecolor{LLyellow}{rgb}{1,1,.8}
\definecolor{darkgreen}{rgb}{0,0.5,0}


%%% \Note, \Chan, and \NoChan are useful prefix commands for the descriptive comments in the column ``Context and Changes''
\newsetbool{boolNeedNewLineBeforeChan}{false} %boolNeedNewLineBeforeChan: true after using \Note; becomes false when using \rowendL
\newcommand{\Note}{\ifbool{boolNeedNewLineBeforeChan}{\newline}{}{\textcolor{blue}{\textbf{-- \textsc{Note:} }}}\setbool{boolNeedNewLineBeforeChan}{true}}
\newcommand{\Chan}{\ifbool{boolNeedNewLineBeforeChan}{\newline}{}\textcolor{mycoral}{{{\bfseries-- {\scshape Changed:} }}}\setbool{boolNeedNewLineBeforeChan}{true}}
\newcommand{\NoChan}{\Chan\ No change.}


\newcommand{\propContrib}{{\ifbool{boolNeedNewLineBeforeChan}{\newline}{}\bfseries\textcolor{blue}{-- Proposed contribution:} }}
\newcommand{\contributors}{\ifbool{boolNeedNewLineBeforeChan}{\newline}{}{\bfseries\textcolor{darkgreen}{-- Contributors:} }\setbool{boolNeedNewLineBeforeChan}{true}}
\newcommand{\submit}{\newline{\bfseries\textcolor{brown}{-- Submission mode:} }}

\newcommand{\ccontext}{{\ifbool{boolNeedNewLineBeforeChan}{\newline}{}\bfseries\textcolor{blue}{-- Context:} }\setbool{boolNeedNewLineBeforeChan}{true}}


\newcommand{\newcol}{&}
\newcommand{\rowendL}{\end{minipage}\\\hline}  %%%NEED TO COORDINATE WITH DEFINITION OF LAST COLUMN, currently using a minipage

\newcommand{\singlebullet}[1]{\protect\begin{itemize}\item #1 \protect\end{itemize}}


\newcounter{cntItA} %counts the items sequentially: 1, 2, ..., 210
\newcounter{cntIssue}\renewcommand{\thecntIssue}{D\arabic{cntIssue}} %counts the reviewer: A, B, ..., L
\newcounter{cntItB}[cntIssue] %counts the item inside the Issue: A1, A2, .., A7, B1, ..., 
% \inc... % increment some counter(s)

%\incItem[label][pdf-bookmark-caption]
\renewcommand{\thecntItB}{\thecntIssue.\arabic{cntItB}}
\newcommandtwoopt{\incItem}[2][][]{%
	\refstepcounter{cntItA}\ifNotEmpty{#1}{\label{it:#1}}\thecntItA \newcol 
	\refstepcounter{cntItB}\ifNotEmpty{#1}{\label{#1}}
	\ifNotEmptyElse{#2}{\def\itemPdfBkmExt{: #2}}{\def\itemPdfBkmExt{}}
	\pdfbookmark[3]{\thecntItB\itemPdfBkmExt}{pdfbkm:\thecntItB}\thecntItB \newcol } %incremment comment number


\newcommand{\rowIssueName}{Issue title}
\newcommand{\needscontributors}{\red{needs contributors}}

%\newcommand{\tabhead}{\rowColorHeader\# & Ref & \centering \scalebox{.75}{\mycol{Location in\\initial draft}} & \centering Comment received & \scalebox{.75}{\mycol{Related\\items}} & \centering Reply notes & \begmin{.04}\scalebox{.8}{\mycol{Revie-\\wed?}}\myendmini}
\setlength{\tabcolsep}{1ex}


%Use of commands defined with optionla arguments inside cells can break the workings of tabular
%See complicated solution in https://tex.stackexchange.com/questions/277140/optional-argument-breaks-command-with-multicolumn
%As a temporary solution, I'll just define all commands as mandatory
%\newcommand{\headreviewer}[2]{\rowcolor{LLyellow}\incIssue[#1]\pdfbookmark[2]{\thecntIssue\ --- #2}{pdfbkm:\thecntIssue}\textbf{\thecntIssue} & Location & Comments from #2 & & & \rowend}


%%% 
%\newcommandtwoopt[2][.320][.320]{\beginlongtabIssue}{%
\newcommand{\beginlongtabIssue}{%
\begin{longtable}{%
|>{\begmin{.025}\centering}c<{\myendmini}%  #
|>{\begmin{.035}\centering}c<{\myendmini}%  Ref
|>{\begmin{.080}}c<{\myendmini}%            Old location
|>{\begmin{.320}}c<{\myendmini}%              Comment / Issue
|>{\begmin{.060}\centering}c<{\myendmini}%  Related
|>{\begmin{.320}}c<{\myendmini}%              Reply notes / Notes about changes
|>{\begmin{.050}\centering}c<{}|%           Rev [in main document]
% The last column needs to be ended explicitly with \rowend, which contains a \end{minipage} followed by \\
}%
}


%%%%%%%%%%%%%%%%%%%%%%%%%%%%%%%%%%%%%%%%%%%%%%%%%%%%%%%%%%%%%%%%%%%%%%%%%%%%%%%%
\newcommand{\headerForIssue}{%
	  \centering \#
	& \centering \scalebox{.7}{Item id}
	& \centering Location %Old location
	& \centering \textbf{Contribution \thecntIssue}: \rowIssueName{}
	& \centering Related %\scalebox{.75}{\mycol{Related\\items}}
	& \centering Context and changes
	& \centering Edit id
}


\newcommand{\githubissue}[1]{\href{https://github.com/zkpstandard/zkreference/issues/#1}{GI#1}}


%%%%%%%%%%%%%%%%%%%%%%%%%%%%%%%%%%%%%%%%%%%%%%%%%%%%%%%%%%%%%%%%
%%%%%%%%%%% TO GENERATE A LIST OF COMMENTS/CONTRIBUTIONS
% contentslines added on each example or examples

\newcommand\listcontributionname{List of Contributions}
\newlistof[chapter]{contribution}{loc}{\listcontributionname}
%creates the file name zkproof-community-reference.contribs with the index of contribution issues/items


%%%%%%%%%%%%%%%%%%%%%%%%%%%%%%%%
%%% Each `issue' is a longtable (can break pages) of comment/contribution items. 
%%% Each `issue' starts with \newIssue and ends with \myendIssue
%%% Besides the automatic header, each row of the table will be an `item' introduced by command \incItem

\newcommand{\newIssue}[2]{%
	\refstepcounter{cntIssue}\label{#1}
	\addcontentsline{loc}{subsection}{Contribution \thecntIssue\ --- #2}
	\renewcommand{\rowIssueName}{#2}
	\noindent\beginlongtabIssue\hline%
			\rowcolor{LLyellow} %must be the first command in the cell, namely before pdfbookmark
			\pdfbookmark[2]{\thecntIssue: \rowIssueName{}}{pdfbkm:\thecntIssue}% adds a pdf bookmark if this is the first header for this Issue	
			\headerForIssue\rowend\hline\hline%
		\endfirsthead\hline%
			\rowcolor{LLgray}\headerForIssue\rowend\hline\hline%
		\endhead%ensures that all lines above this one will repeat as a header
}

\newcommand{\myendIssue}{\end{longtable}}
